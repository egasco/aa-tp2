A partir de los experimentos pudimos confirmar que la variación hiper-parámetros tiene un impacto en la velocidad de aprendizaje de los agentes. A excepción de algunos valores patológicos como puede ser una laza de aprendizaje igual a 1.0 la variación de los mismo tiene un impacto en la velocidad de aprendizaje pero no en la calidad del agente resultante. La calidad de juego del agente esta fuertemente relacionada con como se modela los estados Q. Realizamos varios intentos de mejorar la calidad del agente sin éxito, por ejemplo guardar un estado que tuviese en cuenta el color con el que jugaba el el jugador, pero no se logran mejoras apreciables.  