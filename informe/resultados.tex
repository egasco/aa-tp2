
\subsection{Hiper-parámetros}

Se experimento con diferentes valores iniciales de Q. Los mejores resultados se obtuvieron utilizando 0 como valor inicial, logrando mayor velocidad en el entrenamiento. En la figura ~\ref{fig:qinit} se muestran tasa de victorias de agentes de QLearning contra agentes random.   
\begin{figure}[H]
\begin{center}
\myss{0.3}{{Random_vs_Bot_q_init_0.0}.pdf}{q\_init=0.0}{0.28}
\myss{0.3}{{Random_vs_Bot_q_init_1.0}.pdf}{q\_init=1.0}{0.28}
\myss{0.3}{{Random_vs_Bot_q_init_random}.pdf}{q\_init=random(0,1)}{0.28}
\caption{Velocidad de aprendizaje en función de valor inicial de Q}
\label{fig:qinit}
\end{center}
\end{figure}


En la figura ~\ref{fig:alpha} se compara tasa de victorias para agentes entrenando con diferentes valores para la tasa de aprendizaje. Al utilizar el valor 1.0, que desprecia el conocimiento previo para dicho estado el agente logra una tasa de victorias similar un un agente random. Para todos los valores inferiores a 1.0 se logran una perfomance similar entrenando contra un agente random. La figura ~\ref{fig:QL_alpha_30p_vs_QL_alpha_70p} muestra resultados de 2 agentes que implementan con Qlearning con diferentes tasa de aprendizaje, y en ningún momento uno de los agentes supera significativamente al otro. 
 
\begin{figure}[H]
\begin{center}
\myss{0.35}{alpha}{Bots entrenando con diferentes $\alpha$(tasa de aprendizaje) contra Bot random}{0.30}
\myss{0.35}{QL_alpha_30p_vs_QL_alpha_70p}{Bot $\alpha$=0.3 vs Bot $\alpha$=0.7}{0.30}
\caption{}
\label{fig:alphag}
\end{center}
\end{figure}

Por ultimo experimentamos variando el factor de descuento. Este parámetro controla el peso que tienen los recompensas de estados posteriores. 


\subsection{Exploración vs Explotación}

\begin{figure}[H]
\begin{center}
\myss{0.3}{Bot_epsilon_10p_vs_Bot_epsilon_20p.pdf}{}{0.28}
\myss{0.3}{Bot_epsilon_10p_vs_Bot_epsilon_30p.pdf}{}{0.28}
\myss{0.3}{{Bot_epsilon_10p_vs_Bot_epsilon_40p}.pdf}{}{0.28}
\caption{Bots con diferentes valores de epsilon}
\label{fig:epsilon}
\end{center}
\end{figure}


\begin{figure}[H]
\begin{center}
\myss{0.50}{epsilon.pdf}{}{0.50}
\caption{Bots entrenando con diferentes epsilon contra Bot random}
\label{fig:epsilon2}
\end{center}
\end{figure}


\subsection{Estrategias de entrenamiento}
